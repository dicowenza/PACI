\documentclass{article}

\usepackage[T1]{fontenc} 
\usepackage[utf8]{inputenc}
\usepackage[francais]{babel}
\usepackage{graphicx}


\title{Cahier des charges, projet techonologique}
\author{Mr. Nako}

\begin{document}
\maketitle

\tableofcontents

\begin{abstract}
Ceci est le résumé de l'article
\end{abstract}

\section{Première section}
\subsection{Sous-section 1}

\subsubsection{sous-sous section}

\subsubsection{sous-sous section}

\begin{itemize}
\item item 1
\item item 2
\item item 3
\end{itemize}

\begin{description}
\item[description] ceci est une petite descritption.
\end{description}

\begin{enumerate}
\item salut
\item sa va?
\end{enumerate}

\subsection{Sous-section 2}
\section{Résultats}
\label{resultats}
  ceci est un resultat.
  en voila un autre.

  \newcommand{\paire}[2]{Salut #1 je paire bien? oh oui #2! }

  \paire{1}{2}, \paire{2}{3}
\newcounter{compteur}
  \setcounter{compteur}{1}
  \newenvironment{question}{\noindent{\large \textbf{\paire{0}{4} Question \thecompteur \addtocounter{compteur}{1}}}}{Fin\\[.2cm]}
  
  \begin{question}
  Sa va?
  \end{question}
  après une petite pause et une réponse ...\\
  \begin{question}
    J'ai pas compris, recommence?
    \end{question}
  
\section{Terminus section}
les resultats sont à la section~\ref{resultats}.

Voici quelques formules mathématiques:

$z_i = \sqrt{x^2_1+y^2_1}$
puis
$T(I) = \sum_{i=0}^{I}\frac{h(I)}{M}$
et enfin
\[X = \frac{\int_{\lambda min}^{\lambda max} E(\lambda)S(\lambda)\overline{x}(\lambda)d\lambda}{\int_{\lambda min}^{\lambda max} S(\lambda)\overline{y}(\lambda)d\lambda}\]

\cite{diday-1982}
\cite{preparata-1988}

\bibliographystyle{alpha}
\bibliography{biblio}

\end{document}

